\documentclass{article}
\usepackage[utf8]{inputenc}
\usepackage[margin=1.0in]{geometry}
\usepackage{amsmath}
\usepackage{amsfonts}
\usepackage{amssymb}
\usepackage{enumitem}                       % custom enum labels
\usepackage{parskip}          
\usepackage{physics}
\usepackage{geometry} 
\usepackage{esint}
\geometry{
  paperwidth=18cm,
  left=8mm,
  right=8mm,
  top=8mm,
  bottom=8mm,
}
\usepackage[framemethod=TikZ]{mdframed}     % graphics and framed envs

\renewcommand{\familydefault}{\sfdefault}      % sans serifs text
\setlength{\parindent}{0pt}                    % no paragraph indentation

% region TITLE CONSTRUCTION
\newlength\mywidth
\mywidth=\wd0
\renewcommand{\contentsname}{\hangindent=\mywidth \courseid: \coursetitle \\ \medskip \LARGE{[MY NAME HERE], \semester}}
% endregion

% region FRAMED ENVIRONMENTS
\newcounter{chapter}\setcounter{chapter}{1}
\newcounter{theo}[chapter]\setcounter{theo}{0}
\newcommand{\numTheo}{\arabic{chapter}.\arabic{theo}}
\newcommand{\mdftheo}[3]{
    \mdfsetup{
        frametitle={
            \tikz[baseline=(current bounding box.east),outer sep=0pt]
            \node[anchor=east,rectangle,fill=#3]
            {\ifstrempty{#2}{\strut #1~\numTheo}{\strut #1~\numTheo:~#2}};
        },
        innertopmargin=4pt,linecolor=#3,linewidth=2pt,
        frametitleaboveskip=\dimexpr-\ht\strutbox\relax,
        skipabove=11pt,skipbelow=0pt
    }
}
\newcommand{\mdfnontheo}[3]{
    \mdfsetup{
        frametitle={
            \tikz[baseline=(current bounding box.east),outer sep=0pt]
            \node[anchor=east,rectangle,fill=#3]
            {\ifstrempty{#2}{\strut #1}{\strut #1:~#2}};
        },
        innertopmargin=4pt,linecolor=#3,linewidth=2pt,
        frametitleaboveskip=\dimexpr-\ht\strutbox\relax,
        skipabove=11pt,skipbelow=0pt
    }
}
\newcommand{\mdfproof}[1]{
    \mdfsetup{
        skipabove=11pt,skipbelow=0pt,
        innertopmargin=4pt,innerbottommargin=4pt,
        topline=false,rightline=false,
        linecolor=#1,linewidth=2pt
    }
}
\newenvironment{theorem}[1][]{
    \refstepcounter{theo}
    \mdftheo{Theorem}{#1}{red!25}
    \begin{mdframed}[]\relax
}{\end{mdframed}}

\newenvironment{lemma}[1][]{
    \refstepcounter{theo}
    \mdftheo{Lemma}{#1}{red!15}
    \begin{mdframed}[]\relax
}{\end{mdframed}}

\newenvironment{corollary}[1][]{
    \refstepcounter{theo}
    \mdftheo{Corollary}{#1}{red!15}
    \begin{mdframed}[]\relax
}{\end{mdframed}}

\newenvironment{definition}[1][]{
    \mdfnontheo{Definition}{#1}{blue!20}
    \begin{mdframed}[]\relax
}{\end{mdframed}}

\newenvironment{example}[1][]{
    \mdfnontheo{Example}{#1}{yellow!40}
    \begin{mdframed}[]\relax
}{\end{mdframed}}

\newenvironment{proof}[1][]{
    \mdfproof{black!15}
    \begin{mdframed}[]\relax
\textit{Proof. }}{\end{mdframed}}
% endregion

% region NEW COMMANDS
\newcommand{\ds}{\displaystyle}
\newcommand{\pfn}[1]{\textrm{#1}}  % enables new functions
\newcommand{\mbf}[1]{\mathbf{#1}}  % mathbf
\newcommand{\C}{\mathbb{C}}        % fancy C
\newcommand{\R}{\mathbb{R}}        % fancy R
\newcommand{\Q}{\mathbb{Q}}        % fancy Q
\newcommand{\Z}{\mathbb{Z}}        % fancy Z
\newcommand{\N}{\mathbb{N}}   
\newcommand{\K}{\mathbb{K}}  % fancy N
\newcommand{\V}{\mathbf{V}} %vector space 
\newcommand{\0}{\mathbf{0}} %zero vector 
\newcommand{\from}{\leftarrow}
\renewcommand{\i}[1]{\textit{#1}}
\renewcommand{\b}[1]{\textbf{#1}}
\newcommand{\qed}{$\square$}

% endregion

\title{Ordinary Differential Equations}
\author{lonelyneutrino}
\date{August 2025}
\begin{document}
\maketitle
\section{First Order Differential Equations}
\begin{definition}
    First order ODEs are of the form
    \begin{equation}
        \frac{dy}{dx} = f(x, y)
    \end{equation}
    A function $y = \phi(x)$ is a solution on an interval $I$ if $\phi'(x) = f(x, \phi(x))$ for all $x \in I$.
\end{definition}

\begin{example}
    A simple example of a first order ODE is
    \begin{equation}
        \frac{dy}{dx} = 2x
    \end{equation}
    A solution to this equation is $y = x^2 + C$, where $C$ is a constant.
\end{example}

\begin{example}
    \begin{equation}
        \dv{y}{x} = p(x) y
    \end{equation}
    A solution to this equation is $y = C\exp{\int p(x) \, dx}$, where $C$ is a constant.
\end{example}

\begin{definition}
    An initial value problem (IVP) is a differential equation along with a specified value, called the initial condition, that the solution must satisfy at a given point. For example, the IVP for the first order ODE
    \begin{equation}
        \frac{dy}{dx} = f(x, y)
    \end{equation}
    might specify that $y(x_0) = y_0$ for some point $x_0$ in the interval of interest.
\end{definition}

\subsection{First Order Linear ODEs}
\vspace{5pt}
\begin{definition}
    A first order linear ODE is an equation of the form
    \begin{equation}
        \frac{dy}{dx} + p(x) y = g(x)
    \end{equation}
    where $p(x)$ and $g(x)$ are continuous functions on an interval $I$.
\end{definition}

\begin{theorem}[Integrating Factor Method]
    The general solution to the first order linear ODE
    \begin{equation}
        \frac{dy}{dx} + p(x) y = g(x)
    \end{equation}
    is given by
    \begin{equation}
        y = \frac{1}{\mu(x)} \left( \int \mu(x) g(x) \, dx + C \right)
    \end{equation}
    where $\mu(x) = \exp{\int p(x) \, dx}$ is the integrating factor and $C$ is an arbitrary constant.
\end{theorem}

\begin{proof}
    To prove this theorem, we start with the first order linear ODE
    \begin{equation}
        \frac{dy}{dx} + p(x) y = g(x)
    \end{equation}
    We multiply both sides by the integrating factor $\mu(x)$:
    \begin{equation}
        \mu(x) \frac{dy}{dx} + \mu(x) p(x) y = \mu(x) g(x)
    \end{equation}
    If $\mu(x)$ is chosen such that $\mu'(x) = \mu(x) p(x)$, then the left-hand side becomes the derivative of the product $\mu(x) y$. The integrating factor is given by:
    \begin{equation}
        \mu(x) = \exp{\int p(x) \, dx}
    \end{equation}
    The left-hand side can be rewritten as the derivative of a product:
    \begin{equation}
        \frac{d}{dx} \left( \mu(x) y \right) = \mu(x) g(x)
    \end{equation}
    Integrating both sides with respect to $x$ gives:
    \begin{equation}
        \mu(x) y = \int \mu(x) g(x) \, dx + C
    \end{equation}
    Finally, we solve for $y$:
    \begin{equation}
        y = \frac{1}{\mu(x)} \left( \int \mu(x) g(x) \, dx + C \right)
    \end{equation}
\end{proof}

\begin{example}
    Consider the first order linear ODE 
    \begin{equation}
        y' + \frac12 y = \frac12 \exp{x/3}
    \end{equation}
    To solve this ODE, we first identify the integrating factor:
    \begin{equation}
        \mu(x) = \exp{\int \frac12 \, dx} = \exp{\frac12 x}
    \end{equation}
    Multiplying both sides of the ODE by the integrating factor gives:
    \begin{equation}
        \exp{\frac12 x} y' + \frac12 \exp{\frac12 x} y = \frac12 \exp{\frac56 x}
    \end{equation}
    The left-hand side can be rewritten as:
    \begin{equation}
        \frac{d}{dx} \left( \exp{\frac12 x} y \right) = \frac12 \exp{\frac56 x}
    \end{equation}
    Integrating both sides with respect to $x$ gives:
    \begin{equation}
        \exp{\frac12 x} y = \int \frac12 \exp{\frac56 x} \, dx + C
    \end{equation}
    Finally, we solve for $y$:
    \begin{equation}
        y = \exp{-\frac12 x} \left( \int \frac12 \exp{\frac56 x} \, dx + C \right) 
    \end{equation}
    \begin{equation}
        y = \frac{3}{5} \exp{\frac12 x} + C\exp{-\frac12 x}
    \end{equation}
\end{example}

\subsection{Separable ODEs}
\vspace{5pt}
\begin{definition}
    A separable first order differential equation is of the form 
    $$
    \dv{y}{x} = f(x) g(y)
    $$
    where $f(x)$ and $g(y)$ are functions of $x$ and $y$, respectively.
\end{definition}

\begin{definition}
    A homogeneous first order differential equation is of the form
    $$
    \dv{y}{x} = f\left(\frac{y}{x}\right)
    $$
    where $f$ is a function of the single variable $\frac{y}{x}$.
\end{definition}
A homogeneous equation is solved by making the substitution $v = \frac{y}{x}$, which transforms the equation into a separable form.
\begin{example}
    $$
    \dv{y}{x} = \frac{x^2 - 3y^2}{2xy}
    $$
    Let $v = \frac{y}{x}$, then $y = vx$ and $\dv{y}{x} = v + x\dv{v}{x}$. Substituting these into the original equation gives:
    $$
    v + x\dv{v}{x} = \frac{x^2 - 3(vx)^2}{2x(vx)}
    $$
\end{example}
\end{document}
