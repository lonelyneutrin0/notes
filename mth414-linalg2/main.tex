\documentclass{article}
\usepackage[utf8]{inputenc}
\usepackage[margin=1.0in]{geometry}
\usepackage{amsmath}
\usepackage{amsfonts}
\usepackage{amssymb}
\usepackage{enumitem}                       % custom enum labels
\usepackage{parskip}          
\usepackage{physics}
\usepackage{geometry} 
\usepackage{esint}
\geometry{
  paperwidth=18cm,
  left=8mm,
  right=8mm,
  top=8mm,
  bottom=8mm,
}
\usepackage[framemethod=TikZ]{mdframed}     % graphics and framed envs

\renewcommand{\familydefault}{\sfdefault}      % sans serifs text
\setlength{\parindent}{0pt}                    % no paragraph indentation

% region TITLE CONSTRUCTION
\newlength\mywidth
\mywidth=\wd0
\renewcommand{\contentsname}{\hangindent=\mywidth \courseid: \coursetitle \\ \medskip \LARGE{[MY NAME HERE], \semester}}
% endregion

% region FRAMED ENVIRONMENTS
\newcounter{chapter}\setcounter{chapter}{1}
\newcounter{theo}[chapter]\setcounter{theo}{0}
\newcommand{\numTheo}{\arabic{chapter}.\arabic{theo}}
\newcommand{\mdftheo}[3]{
    \mdfsetup{
        frametitle={
            \tikz[baseline=(current bounding box.east),outer sep=0pt]
            \node[anchor=east,rectangle,fill=#3]
            {\ifstrempty{#2}{\strut #1~\numTheo}{\strut #1~\numTheo:~#2}};
        },
        innertopmargin=4pt,linecolor=#3,linewidth=2pt,
        frametitleaboveskip=\dimexpr-\ht\strutbox\relax,
        skipabove=11pt,skipbelow=0pt
    }
}
\newcommand{\mdfnontheo}[3]{
    \mdfsetup{
        frametitle={
            \tikz[baseline=(current bounding box.east),outer sep=0pt]
            \node[anchor=east,rectangle,fill=#3]
            {\ifstrempty{#2}{\strut #1}{\strut #1:~#2}};
        },
        innertopmargin=4pt,linecolor=#3,linewidth=2pt,
        frametitleaboveskip=\dimexpr-\ht\strutbox\relax,
        skipabove=11pt,skipbelow=0pt
    }
}
\newcommand{\mdfproof}[1]{
    \mdfsetup{
        skipabove=11pt,skipbelow=0pt,
        innertopmargin=4pt,innerbottommargin=4pt,
        topline=false,rightline=false,
        linecolor=#1,linewidth=2pt
    }
}
\newenvironment{theorem}[1][]{
    \refstepcounter{theo}
    \mdftheo{Theorem}{#1}{red!25}
    \begin{mdframed}[]\relax
}{\end{mdframed}}

\newenvironment{lemma}[1][]{
    \refstepcounter{theo}
    \mdftheo{Lemma}{#1}{red!15}
    \begin{mdframed}[]\relax
}{\end{mdframed}}

\newenvironment{corollary}[1][]{
    \refstepcounter{theo}
    \mdftheo{Corollary}{#1}{red!15}
    \begin{mdframed}[]\relax
}{\end{mdframed}}

\newenvironment{definition}[1][]{
    \mdfnontheo{Definition}{#1}{blue!20}
    \begin{mdframed}[]\relax
}{\end{mdframed}}

\newenvironment{example}[1][]{
    \mdfnontheo{Example}{#1}{yellow!40}
    \begin{mdframed}[]\relax
}{\end{mdframed}}

\newenvironment{proof}[1][]{
    \mdfproof{black!15}
    \begin{mdframed}[]\relax
\textit{Proof. }}{\end{mdframed}}
% endregion

% region NEW COMMANDS
\newcommand{\ds}{\displaystyle}
\newcommand{\pfn}[1]{\textrm{#1}}  % enables new functions
\newcommand{\mbf}[1]{\mathbf{#1}}  % mathbf
\newcommand{\C}{\mathbb{C}}        % fancy C
\newcommand{\R}{\mathbb{R}}        % fancy R
\newcommand{\Q}{\mathbb{Q}}        % fancy Q
\newcommand{\Z}{\mathbb{Z}}        % fancy Z
\newcommand{\N}{\mathbb{N}}   
\newcommand{\K}{\mathbb{K}}  % fancy N
\newcommand{\V}{\mathbf{V}} %vector space 
\newcommand{\0}{\mathbf{0}} %zero vector 
\newcommand{\from}{\leftarrow}
\renewcommand{\i}[1]{\textit{#1}}
\renewcommand{\b}[1]{\textbf{#1}}
\newcommand{\qed}{$\square$}
\newcommand{\range}{\text{range }}

% endregion

\title{Linear Algebra II}
\author{Hrishikesh Belagali}
\date{August 2025}
\begin{document}
\maketitle
\section{Basics}
\begin{example}
    For a set $\mathcal S$, let $\mathbb F^{\mathcal S}$ be 
    the set of all functions from $\mathcal S$ to $\mathbb F$.
    Then, defined over canonical addition and scalar
    multiplications, $\mathbb F^{\mathcal S}$ is a vector
    space. The additive identity is the zero function $0$,
    defined as $0(x) = 0$. The additive inverse can be defined
    as $-f : \mathcal S \to \mathbb F$ defined as $-f(x) = 
    -(f(x)) \forall x \in \mathcal S$. \\
    
    Note that $\mathbb F^n$ and $\mathbb F^{\infty}$ are 
    special cases of $\mathbb F^{\mathcal S}$, 
    where $\mathcal S$ is a finite set of size $n$ or 
    an infinite set, respectively.
\end{example}

Note that the empty set $\phi$ is not a vector space, nor is it a subspace of any vector space.
\begin{example}
The set of differentiable real-valued functions is a subspace of $\mathbb R^{\mathbb R}$. Note that in calculus, the sum of two continuous functions 
is continuous, and the sum of two differentiable functions is differentiable. Also, scalar multiples of continuous and differentiable functions are continuous and differentiable, respectively.
\end{example}

\begin{definition}
    Let $V_1, \cdots V_n$ be subspaces of a vector space $\V$. Then, the sum of these subspaces is defined as 
    $$V_1 + V_2 + \cdots + V_n = \{ v_1 + v_2 + \cdots + v_n \mid v_i \in V_i \text{ for all } i \}$$
\end{definition}

\begin{example}
    Let $$V_1 = \left \{ (w, w, x, x) \in \mathbb F^4 | w, x \in \mathbb F \right \}$$
    $$V_2 = \left \{ (y, y, y, z) \in \mathbb F^4 | y, z \in \mathbb F \right \}$$
    Now, let $v_1 \in V_1$ and $v_2 \in V_2$. Then, we can write
    $$v_1 = (w_1, w_1, x_1, x_1)$$
    $$v_2 = (y_2, y_2, y_2, z_2)$$
    for some $w_1, x_1, y_2, z_2 \in \mathbb F$. Then, we have
    $$v_1 + v_2 = (w_1 + y_2, w_1 + y_2, x_1 + y_2, x_1 + z_2) \in V_1 + V_2$$
    Let $W$ be defined as 
    $$W = \left \{ (x, x, y, z) \in \mathbb F^4 | x, y, z, \in \mathbb F \right \}$$
    Then, $v_1 + v_2 \in W$ so $V_1 + V_2 \subseteq W$. \\
    Let $w \in W$. Then, we can write
    $$w = (x_w, x_w, y_w, z_w)$$
    for some $x_w, y_w, z_w \in \mathbb F$. Then, we have
    $$w = (x_w, x_w, y_w, z_w) = (x_w, x_w, y_w, y_w) + (0, 0, 0, z_w - y_w) \in V_1 + V_2$$
    $\therefore W = V_1 + V_2$
\end{example}

\begin{lemma}
    For any subspaces $V_1, \cdots V_n$ of a vector space $\V$, $V_1 + \cdots + V_n$ is a subspace of $\V$. It is also the smallest subspace of $V$
    that contains all elements of the form $v_1 + \cdots + v_n$ where $v_i \in V_i$ for all $i$.
\end{lemma}
\begin{proof}
    From the definition and that $V_1, \cdots V_n$ are subspaces,
    Since the subspaces themselves are closed under addition and scalar multiplication, $V_1 + \cdots + V_n$ is also closed under addition and scalar multiplication. Also, the zero vector $\0$ is in each of the subspaces, so $\0 \in V_1 + \cdots + V_n$. Thus, $V_1 + \cdots + V_n$ is a subspace of $\V$. 
\end{proof}

Note: Generally, the set theoretic union is rarely a subspace, except for trivial cases where one space is a subspace of the other. 
However, intersections of subspaces are generally subspaces.

\begin{definition}[Direct sum]
    Let $V_1, \cdots V_n$ be subspaces of a vector space $\V$. Then, the sum $V_1 + \cdots + V_n$ is called a direct sum if each element of $V_1 + \cdots + V_n$ can be written in one and only one way as $v_1 + \cdots + v_n$ where $v_i \in V_i$ for all $i$. 
    In this case, we say that the sum is a direct sum, denoted by 
    $$W = V_1 \oplus V_2 \oplus \cdots \oplus V_n$$
\end{definition}

\begin{example}
    Let $$U = \left \{ (x, x, y) \in \mathbb F^3 | x, y \in \mathbb F \right \}$$
    Let $$W = \left \{ (x, 0, 0) \in \mathbb F^3 | x \in \mathbb F \right \}$$
    Then, $U$ and $W$ are subspaces of $\mathbb F^3$.
    Any arbitrary vector in $\mathbb F^3$ can be written as
    $$\begin{pmatrix}
        a \\
        b \\
        c
    \end{pmatrix} = \begin{pmatrix}
        b \\
        b \\
        c
    \end{pmatrix} + \begin{pmatrix}
        a-b \\
        0 \\
        0
    \end{pmatrix}$$
    Since this is a unique representation, $U \oplus W = \mathbb F^3$.
\end{example}
\begin{theorem}
    Let $V_1, \cdots V_n$ be subspaces of a vector space $\V$. Then, $V_1  + \cdots V_n = V_1 \oplus \cdots \oplus V_n$ if and only if the only way to write $\0$ as $v_1 + \cdots + v_n$ where $v_i \in V_i$ for all $i$ is to take each $v_i = \0$.
    In other words, if $v_1 + \cdots + v_n = \0$ implies that each $v_i = \0$, then the sum is a direct sum.
\end{theorem}
\begin{proof}
    Suppose that $V_1 + \cdots V_n$ is a direct sum. Then, the additive identity can be written as 
    the sum of additive identities from each subspace. By definition of a direct sum, this is the \textit{only}
    way to write the additive identity as a sum. \\ 
    Suppose that the only way to write zero is as the sum 
    of additive identities from each subspace. Consider an arbitrary vector 
    $v \in V$. Suppose that there are two different ways of writing the sum,
    $$v = u_1 + \cdots u_n ; u_k \in V_k$$ 
    $$v = v_1 + \cdots v_n ; v_k \in V_k$$
    Then, we can subtract these two equations 
    $$0 = (u_1 - v_1) + \cdots + (u_n - v_n) ; (u_k - v_k) \in V_k$$
    Since the only way to write zero is as the sum of additive identities from each subspace, we must have $u_k - v_k = 0$ for all $k$. Thus, $u_k = v_k$ for all $k$, and the representation is unique.
    Therefore, by definition, the sum $V_1 + \cdots V_n$ is a direct sum.
\end{proof}

\begin{theorem}
    Let $U$ and $W$ be subspaces of a vector space $\V$. Then, the sum $U + W$ is a direct sum if and only if $U \cap W = \{0\}$.
\end{theorem}
\begin{proof}
    Suppose that $U + W$ is a direct sum. Let $v \in U \cap W$. Then, $v \in U$, and $-v \in W$
    $$0 = v + (-v)$$
    Since the representation is unique, we must have $v = 0$. Thus, $U \cap W = \{0\}$.

    Conversely, suppose that $U \cap W = \{0\}$. Let $u \in U$ and $w \in W$. Then, we can write
    $$u + w = 0$$
    From the previous result, it suffices to show that $u = w = 0$. This implies that $w$ is the additive inverse
    of $u$, meaning $u, w \in U \cap W = \{0\}$. Therefore, $u = w = 0$ and $U + W$ = $U \oplus W$.
\end{proof}
\begin{lemma}
    Suppose $v_1, \cdots v_m$ is a linearly dependent list in V. Then there exists $j \in \{1, \cdots m\}$ such that
    $$v_j \in \text{span}(v_1, \cdots, v_{j-1})$$
    If this condition holds, then $$\text{span} (v_1, \cdots v_m) = \text{span} (v_1, \cdots, v_{j-1}, v_{j+1}, \cdots v_m)$$
\end{lemma}
\begin{proof}
    Since $v_1, \cdots v_m$ is linearly dependent,
    there exist $a_1, \cdots a_m \in \mathbb F$ not all zero
    such that
    $$a_1v_1 + \cdots + a_mv_m = 0$$
    Let $k$ be the largest index such that $a_k \neq 0$.
    Then, $$v_k = -\frac{a_1}{a_k}v_1 - \cdots - \frac{a_{k-1}}{a_k}v_{k-1}$$
    Therefore, $v_k \in \text{span}(v_1, \cdots, v_{k-1})$. \\
    \\Suppose $v_k \in \text{span}(v_1, \cdots, v_{k-1})$. Then, we can write
    $$v_k = b_1v_1 + \cdots + b_{k-1}v_{k-1}$$
    for some $b_1, \cdots, b_{k-1} \in \mathbb F$.
    Let $u \in \text{span}(v_1, \cdots v_m)$. Then, 
    $$u = c_1v_1 + \cdots c_mv_m \text{ for some } c_1, \cdots c_m \in \mathbb F$$
    Substituting for $v_k$, we get
    $$u = c_1v_1 + \cdots + c_{k-1}v_{k-1} + c_k(b_1v_1 + \cdots + b_{k-1}v_{k-1}) + c_{k+1}v_{k+1} + \cdots + c_mv_m$$
    Therefore, 
    $u \in \text{span}(v_1, \cdots, v_{k-1}, v_{k+1}, \cdots v_m)$.
    Proving the other direction is trivial, you can set the coefficients of higher indices than $k$ to zero.
\end{proof}
\begin{theorem}
    Let $V$ be a finite-dimensional vector space. Suppose that 
    $u_1, \cdots u_m$ is linearly independent in $V$ and $w_1, \cdots w_n$.
    Then, $m \leq n$. In other words, any linearly independent list is smaller
    or the same size as any spanning list.
\end{theorem}

\begin{theorem}
    Suppose $U$ is a subspace of a finite-dimensional vector space $V$. Then there exists
    a subspace $W$ of $V$ such that $V = U \oplus W$.
\end{theorem}
\begin{proof}
    Let $u_1, \cdots u_m$ be a basis of $U$. Since $U$ is a subspace of $V$, we can extend this basis to a basis of $V$, say $u_1, \cdots u_m, w_1, \cdots w_n$. Let $W = \text{span}(w_1, \cdots w_n)$. Then, we have
    $$V = U + W$$
    since any vector in $V$ can be written as a linear combination of the basis vectors. \\
    Now, we need to show that the sum is direct. Suppose that 
    $$u + w = 0 ; u \in U, w \in W$$
    Then, we can write
    $$u = a_1u_1 + \cdots + a_mu_m ; a_i \in \mathbb F$$
    $$w = b_1w_1 + \cdots + b_nw_n ; b_i \in \mathbb F$$
    Therefore, we have
    $$a_1u_1 + \cdots + a_mu_m + b_1w_1 + \cdots + b_nw_n = 0$$
    Since the basis vectors are linearly independent, all coefficients must be zero.
    Thus, $u = w = 0$, and the sum is direct.
\end{proof}
\begin{theorem}
    If $V_1$ and $V_2$ are finite-dimensional subspaces of $V$,
    then 
    $$\text{dim}(V_1 + V_2) = \text{dim}(V_1) + \text{dim}(V_2) - \text{dim}(V_1 \cap V_2)$$
\end{theorem}
\begin{proof}
    Let $u_1, \cdots u_m$ be a basis of $V_1 \cap V_2$. We can extend this basis to a basis of $V_1$, say $u_1, \cdots u_m, v_1, \cdots v_k$. Similarly, we can extend the basis of $V_1 \cap V_2$ to a basis of $V_2$, say $u_1, \cdots u_m, w_1, \cdots w_l$. We claim that the list 
    $$u_1, \cdots u_m, v_1, \cdots v_k, w_1, \cdots w_l$$
    is a basis of $V_1 + V_2$. \\
    Since the dimension of $V_1$ is $m+k$ and the dimension of $V_2$ is $m+l$, the dimension of $V_1 + V_2$ is at most $m+k+l$
    Therefore,
    $$\dim (V_1 + V_2) = m + k + l = (m+k) + (m+l) - m = \dim(V_1) + \dim(V_2) - \dim(V_1 \cap V_2)$$
    It remains to show that the list spans $V_1 + V_2$ and is linearly independent. \\
    Let $v \in V_1 + V_2$. Then, we can write 
    $$v = v_1 + v_2 ; v_1 \in V_1, v_2 \in V_2$$
    Since $u_1, \cdots u_m, v_1, \cdots v_k$ is a basis of $V_1$, we can write
    $$v_1 = a_1u_1 + \cdots + a_mu_m + b_1v_1 + \cdots + b_kv_k ; a_i, b_i \in \mathbb F$$
    Similarly, we can write
    $$v_2 = c_1u_1 + \cdots + c_mu_m + d_1v_1 + \cdots + d_kv_k ; c_i, d_i \in \mathbb F$$  
    Therefore, we have
    $$v = (a_1 + c_1)u_1 + \cdots + (a_m + c_m)u_m + b_1v_1 + \cdots + b_kv_k + d_1w_1 + \cdots + d_lw_l$$
    Thus, the list spans $V_1 + V_2$. \\
    To show that the list is linearly independent, suppose we have a linear combination
    $$\sum_{i=1}^m \alpha_i u_i + \sum_{j=1}^k \beta_j v_j + \sum_{l=1}^l \gamma_l w_l = 0$$
    for some scalars $\alpha_i, \beta_j, \gamma_l \in \mathbb F$. We need to show that all coefficients must be zero. \\
    Since $u_1, \cdots u_m, v_1, \cdots v_k$ is a basis of $V_1$, we can write
    $$v_1 = a_1u_1 + \cdots + a_mu_m + b_1v_1 + \cdots + b_kv_k ; a_i, b_i \in \mathbb F$$
    Similarly, we can write
    $$v_2 = c_1u_1 + \cdots + c_mu_m + d_1v_1 + \cdots + d_kv_k ; c_i, d_i \in \mathbb F$$
    Complete ts later
\end{proof}
\section{Invariant subspaces}
\begin{definition}[Invariant subspace]
    Let $T \in \mathcal L(\V)$. A subspace $U$ of $\V$ is called invariant under $T$ if 
    $$u \in U \implies T(u) \in U$$
    In other words, $U$ is invariant under $T$ if $T(U) \subseteq U$.
\end{definition}
\begin{definition}[Eigenvalues and eigenvectors]
    A number $\lambda \in \mathbb F$ is called an eigenvalue of $T \in \mathcal L(\V)$ if there exists a non-zero vector $v \in \V$ such that
    $$T(v) = \lambda v$$
    Such a vector $v$ is called an eigenvector corresponding to the eigenvalue $\lambda$.
\end{definition}
\begin{definition}[Polynomials of linear operators]
    Let $m$ be a positive integer. Define $T^m$ as $T \circ T \circ \cdots \circ T$ ($m$ times). Define $T^0 = I$. Define $T^{-m} = (T^{-1})^m$ if $T$ is invertible.
    Then, 
    let $p(z) = a_mz^m + a_{m-1}z^{m-1} + \cdots + a_1z + a_0$ be a polynomial with coefficients in $\mathbb F$. Then, we can define the polynomial of the operator $T$ as
    $$p(T) = a_mT^m + a_{m-1}T^{m-1} + \cdots + a_1T + a_0I$$
\end{definition}
\begin{lemma}
If $p$, $q$ are polynomials and $T \in \mathcal L(\mathbf V)$, then $(pq)(T) = p(T)q(T) = q(T)p(T)$.
\end{lemma}
\begin{proof}
    Let $p(z) = a_mz^m + a_{m-1}z^{m-1} + \cdots + a_1z + a_0$ and $q(z) = b_nz^n + b_{n-1}z^{n-1} + \cdots + b_1z + b_0$. Then, we have
    $$p(T) = a_mT^m + a_{m-1}T^{m-1} + \cdots + a_1T + a_0I$$
    $$q(T) = b_nT^n + b_{n-1}T^{n-1} + \cdots + b_1T + b_0I$$
    Now, we can compute the product $p(T)q(T)$ as follows:
    \begin{align*}
        p(T)q(T) &= (a_mT^m + a_{m-1}T^{m-1} + \cdots + a_1T + a_0I)(b_nT^n + b_{n-1}T^{n-1} + \cdots + b_1T + b_0I) \\
        &= a_mb_nT^{m+n} + (a_mb_{n-1} + a_{m-1}b_n)T^{m+n-1} + \cdots + (a_0b_0)I
    \end{align*}
    This is exactly the polynomial $(pq)(T)$, where $(pq)(z) = p(z)q(z)$.
    Since multiplication of polynomials is commutative, we also have $p(T)q(T) = q(T)p(T)$.
    \end{proof}
    \begin{theorem}
    Let $T \in \mathcal L(\V)$ and $p \in \mathcal P(\mathbb F)$. Then, $\text{null } p(T)$ and $\text{range } p(T)$ are invariant under $T$.
    \end{theorem}
    \begin{proof}
        Let $u \in \text{null } p(T)$. Then, we have
        $$p(T)(u) = 0$$
        Applying $T$ to both sides, we get
        $$p(T)(Tu) = (p(T)T)u = (Tp(T))u = T(p(T)u) = T(0) = 0$$
        Thus, $T(u) \in \text{null } p(T)$, showing that $\text{null } p(T)$ is invariant under $T$.

        Now, let $v \in \text{range } p(T)$. Then, we can write
        $$v = p(T)(w)$$
        for some $w \in \V$. Applying $T$, we get
        $$T(v) = T(p(T)(w)) = p(T)(T(w))$$
        Thus, $T(v) \in \text{range } p(T)$, showing that $\text{range } p(T)$ is invariant under $T$.
    \end{proof}
    \subsection{Characteristic polynomials}
    \begin{definition}[Monic polynomial]
        A polynomial $p \in \mathcal P(\mathbb F)$ is called monic if the leading coefficient is 1.
    \end{definition}
    \begin{definition}[Minimal polynomial]
        Let $T \in \mathcal L(\V)$. A monic polynomial $p$ of smallest degree such that $p(T) = 0$ is called the minimal polynomial of $T$.
    \end{definition}
    \begin{theorem}
        Let $\mathbf V$ be finite dimensional. Let $T \in \mathcal L(\mathbf V)$. Then, there is a 
        unique monic polynomial $p$ in $\mathcal P(\mathbb F)$ of smallest degree such that $p(T) = 0$.
        Furthermore, the degree of $\deg p \leq \dim \mathbf V$.
    \end{theorem}
    \begin{proof}\\\\
        \textbf{Existence:} We will prove it by induction on the dimension of $\V$. \\\\
        \textit{Base case:} \\ If $\dim \V = 0$, then $\V = \{0\}$. Thus, $T$ is the zero operator, and we can take $p(z) = 1$. \\\\
        \textit{Inductive step:} \\
        Suppose that $\dim \V > 0$ and suppose that the results is true for all operators
        on all vector spaces of dimension strictly less than $\dim \V$. Let $u \in \V, u \neq 0$
        The list $u, Tu, T^2u, \cdots T^nu$ must be linearly dependent for some $n \leq \dim \V$.
        Then there exists a smallest positive integer $m$ such that
        $T^mu$ is a linear combination of the previous $T^ku$'s. Thus, we can write
        $$T^mu = -a_{m-1}T^{m-1}u - \cdots - a_1Tu - a_0u$$
        for some $a_0, \cdots a_{m-1} \in \mathbb F$. Let
        $$p_1(z) = z^m + a_{m-1}z^{m-1} + \cdots + a_1z + a_0$$
        Then, we have $p_1(T)(u) = 0$. \\
        Since $\text{null } p_1(T)$ is invariant under $T$, $u, Tu, T^2u, \cdots T^{m-1}u \in \text{null } p_1(T)$.
        Note that from the definition of $m$, the list $u, Tu, T^2u, \cdots T^{m-1}u$ is linearly independent.
        Therefore, $\dim \text{null } p_1(T) \geq m$. Then, 
        $$\dim \text{range } p_1(T) = \dim \V - \dim \text{null } p_1(T) \leq \dim \V - m < \dim \V \leq n - m$$
        Since $\text{range } p_1(T)$ is invariant under $T$, we can apply the inductive hypothesis to the operator
        $T|_{\text{range } p_1(T)}$. Thus, there exists a unique monic polynomial $q \in \mathcal P(\mathbb F)$ of degree $\leq n - m$
        such that $q(T|_{\text{range } p_1(T)}) = 0$. Then, for all $v \in \V$, we have 
        $p_1(T) \in \text{ range}q(T)$. Thus, $q(T)(p_1(T)(v)) = 0$. Therefore, we have
        $$(q p_1)(T)(v) = q(T)(p_1(T)(v)) = 0$$
        for all $v \in \V$. Therefore, $(qp_1)(T) = 0$, and we can take $p = qp_1$ as our minimal polynomial.
        
        \textbf{Uniqueness:} Suppose $p$ and $q$ are both monic polynomials of smallest degree
        such that $p(T) = q(T) = 0$. Then $(p-q)(T) = 0$, and since $p$ and $q$ have the
        same degree and leading coefficient, $p-q$ has degree strictly less than $\deg p$. Note that 
        $p-q$ is not necessarily monic. However, we can write $p-q = c r$ where $c$ is the leading coefficient of $p-q$ and $r$ is a monic polynomial. Then, we have
        $$0 = (p-q)(T) = c r(T)$$
        By minimality of $\deg p$, we must have $p-q = 0$, so $p = q$.
    \end{proof}
\begin{definition}
    Let $\V$ be finite-dimensional and $T \in \mathcal L(\V)$. Let $p(z)$ be the 
    minimal polynomial of $T$. Then, 
    \begin{itemize}
        \item The zeros of $p$ are the eigenvalues of $T$.
        \item If $\V$ is a complex vector space, then 
        $$p(z) = (z - \lambda_1)(z - \lambda_2) \cdots (z - \lambda_m)$$
        where $\lambda_1, \cdots \lambda_m$ are the eigenvalues of $T$ (possibly repeated).
    \end{itemize}
\end{definition}
\begin{proof}
    \textbf{Forward: If $\lambda$ is a zero of $p$, then $\lambda$ is an eigenvalue of $T$.} \\
    Suppose that $p(\lambda) = 0$. Then, we can write
    $$p(z) = (z - \lambda)q(z)$$
    where $q(z)$ is a monic polynomial.
    Then, we have 
    $$p(T) = (T - \lambda I)q(T) = 0$$
    For some $v \in \V$ such that $q(T)v \neq 0$, we have 
    $$(T - \lambda I)(q(T)v) = 0$$
    Since $\deg{q} = \deg{p} - 1 < \deg{p}$, such a $v$ exists.
    Thus, $T(q(T)v) = \lambda (q(T)v)$, and $q(T)v$ is an eigenvector corresponding to the eigenvalue $\lambda$. \\\\
    \textbf{Reverse: If $\lambda$ is an eigenvalue of $T$, then $\lambda$ is a zero of $p$.} \\
    Suppose that $\lambda$ is an eigenvalue of $T$. Then, there exists a non-zero vector $v \in \V$ such that
    $$T(v) = \lambda v$$
    Then, we have
    $$T(Tv) = T(\lambda v) = \lambda T(v) = \lambda^2 v$$
    Continuing this way, we can show that
    $$T^k(v) = \lambda^k v$$
    for all non-negative integers $k$. Then, we have
    $$p(T)(v) = (T^m + a_{m-1}T^{m-1} + \cdots + a_1T + a_0I)(v) = ( \lambda^m + a_{m-1}\lambda^{m-1} + \cdots + a_1\lambda + a_0)v = p(\lambda)v$$
    Since $p(T) = 0$, we have $p(\lambda)v = 0$. Since $v \neq 0$, we must have $p(\lambda) = 0$.

    By the fundamental theorem of algebra, if $\V$ is a complex vector space, then $p(z)$ can be factored into linear factors. Since the zeros of $p$ are the eigenvalues of $T$, we can write
    $$p(z) = (z - \lambda_1)(z - \lambda_2) \cdots (z - \lambda_m)$$
    where $\lambda_1, \cdots \lambda_m$ are the eigenvalues of $T$ (possibly repeated).
\end{proof}
\begin{corollary}
    Let $\V$ be a nonzero, finite-dimensional, complex vector space and $T \in \mathcal L(\V)$.
    Then, $T$ has an eigenvalue.
\end{corollary}
Note: The theorem is not true for infinite-dimensional vector spaces.
For example, consider the vector space $\mathbb C^{\infty}$ and the right shift operator $T$ defined as
$$T(x_1, x_2, x_3, \cdots) = (0, x_1, x_2, x_3, \cdots)$$
Then, $T$ has no eigenvalues. To see this, suppose that $\lambda$ is an eigenvalue of $T$. Then, there exists a non-zero vector $v = (x_1, x_2, x_3, \cdots)$ such that
$$T(v) = \lambda v$$
This gives us the equation
$$(0, x_1, x_2, x_3, \cdots) = \lambda (x_1, x_2, x_3, \cdots)$$
Equating components, we find that
$$0 = \lambda x_1$$
$$x_1 = \lambda x_2$$
$$x_2 = \lambda x_3$$
$$\vdots$$
If $\lambda \neq 0$, this implies that $x_1 = x_2 = x_3 = \cdots = 0$, contradicting the assumption that $v$ is non-zero. Therefore, we must have $\lambda = 0$. However, if $\lambda = 0$, then we have
$$T(v) = (0, x_1, x_2, x_3, \cdots) = 0$$
which again implies that $v = 0$. Thus, we conclude that $T$ has no eigenvalues. \\\\ 
Recall that $q(z)$ divides $p(z)$ if there exists a polynomial $r(z)$ such that $p(z) = q(z)r(z)$.
\begin{theorem}
    Let $\V$ be finite-dimensional and $T \in \mathcal L(\V)$.
    Let $p(z)$ be the minimal polynomial of $T$.
    Let $q(z) \in \mathcal P(\mathbb F)$.  Then, $q(T) = 0$ if and only if $p(z)$ divides $q(z)$.
\end{theorem}
\begin{proof}
    If $q = ps$ for some $s \in \mathcal P(\mathbb F)$, then
    $q(T) = p(T)s(T) = 0$, $s(T) = 0$. \\
    Conversely, suppose that $q(T) = 0$. Using polynomial long division, we can write
    $$q(z) = p(z)s(z) + r(z)$$
    where $\deg r < \deg p$. Then, we have
    $$r(T) = q(T) - p(T)s(T) = 0 - 0 = 0$$
    Since $\deg r < \deg p$, we must have $r(z) = 0$. Thus, we conclude that $p(z)$ divides $q(z)$.
\end{proof}
\end{document}