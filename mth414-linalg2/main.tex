\documentclass{article}
\usepackage[utf8]{inputenc}
\usepackage[margin=1.0in]{geometry}
\usepackage{amsmath}
\usepackage{amsfonts}
\usepackage{amssymb}
\usepackage{enumitem}                       % custom enum labels
\usepackage{parskip}          
\usepackage{physics}
\usepackage{geometry} 
\usepackage{esint}
\geometry{
  paperwidth=18cm,
  left=8mm,
  right=8mm,
  top=8mm,
  bottom=8mm,
}
\usepackage[framemethod=TikZ]{mdframed}     % graphics and framed envs

\renewcommand{\familydefault}{\sfdefault}      % sans serifs text
\setlength{\parindent}{0pt}                    % no paragraph indentation

% region TITLE CONSTRUCTION
\newlength\mywidth
\mywidth=\wd0
\renewcommand{\contentsname}{\hangindent=\mywidth \courseid: \coursetitle \\ \medskip \LARGE{[MY NAME HERE], \semester}}
% endregion

% region FRAMED ENVIRONMENTS
\newcounter{chapter}\setcounter{chapter}{1}
\newcounter{theo}[chapter]\setcounter{theo}{0}
\newcommand{\numTheo}{\arabic{chapter}.\arabic{theo}}
\newcommand{\mdftheo}[3]{
    \mdfsetup{
        frametitle={
            \tikz[baseline=(current bounding box.east),outer sep=0pt]
            \node[anchor=east,rectangle,fill=#3]
            {\ifstrempty{#2}{\strut #1~\numTheo}{\strut #1~\numTheo:~#2}};
        },
        innertopmargin=4pt,linecolor=#3,linewidth=2pt,
        frametitleaboveskip=\dimexpr-\ht\strutbox\relax,
        skipabove=11pt,skipbelow=0pt
    }
}
\newcommand{\mdfnontheo}[3]{
    \mdfsetup{
        frametitle={
            \tikz[baseline=(current bounding box.east),outer sep=0pt]
            \node[anchor=east,rectangle,fill=#3]
            {\ifstrempty{#2}{\strut #1}{\strut #1:~#2}};
        },
        innertopmargin=4pt,linecolor=#3,linewidth=2pt,
        frametitleaboveskip=\dimexpr-\ht\strutbox\relax,
        skipabove=11pt,skipbelow=0pt
    }
}
\newcommand{\mdfproof}[1]{
    \mdfsetup{
        skipabove=11pt,skipbelow=0pt,
        innertopmargin=4pt,innerbottommargin=4pt,
        topline=false,rightline=false,
        linecolor=#1,linewidth=2pt
    }
}
\newenvironment{theorem}[1][]{
    \refstepcounter{theo}
    \mdftheo{Theorem}{#1}{red!25}
    \begin{mdframed}[]\relax
}{\end{mdframed}}

\newenvironment{lemma}[1][]{
    \refstepcounter{theo}
    \mdftheo{Lemma}{#1}{red!15}
    \begin{mdframed}[]\relax
}{\end{mdframed}}

\newenvironment{corollary}[1][]{
    \refstepcounter{theo}
    \mdftheo{Corollary}{#1}{red!15}
    \begin{mdframed}[]\relax
}{\end{mdframed}}

\newenvironment{definition}[1][]{
    \mdfnontheo{Definition}{#1}{blue!20}
    \begin{mdframed}[]\relax
}{\end{mdframed}}

\newenvironment{example}[1][]{
    \mdfnontheo{Example}{#1}{yellow!40}
    \begin{mdframed}[]\relax
}{\end{mdframed}}

\newenvironment{proof}[1][]{
    \mdfproof{black!15}
    \begin{mdframed}[]\relax
\textit{Proof. }}{\end{mdframed}}
% endregion

% region NEW COMMANDS
\newcommand{\ds}{\displaystyle}
\newcommand{\pfn}[1]{\textrm{#1}}  % enables new functions
\newcommand{\mbf}[1]{\mathbf{#1}}  % mathbf
\newcommand{\C}{\mathbb{C}}        % fancy C
\newcommand{\R}{\mathbb{R}}        % fancy R
\newcommand{\Q}{\mathbb{Q}}        % fancy Q
\newcommand{\Z}{\mathbb{Z}}        % fancy Z
\newcommand{\N}{\mathbb{N}}   
\newcommand{\K}{\mathbb{K}}  % fancy N
\newcommand{\V}{\mathbf{V}} %vector space 
\newcommand{\0}{\mathbf{0}} %zero vector 
\newcommand{\from}{\leftarrow}
\renewcommand{\i}[1]{\textit{#1}}
\renewcommand{\b}[1]{\textbf{#1}}
\newcommand{\qed}{$\square$}

% endregion

\title{Linear Algebra II}
\author{Hrishikesh Belagali}
\date{August 2025}
\begin{document}
\maketitle
\section{Basics}
\begin{example}
    For a set $\mathcal S$, let $\mathbb F^{\mathcal S}$ be 
    the set of all functions from $\mathcal S$ to $\mathbb F$.
    Then, defined over canonical addition and scalar
    multiplications, $\mathbb F^{\mathcal S}$ is a vector
    space. The additive identity is the zero function $0$,
    defined as $0(x) = 0$. The additive inverse can be defined
    as $-f : \mathcal S \to \mathbb F$ defined as $-f(x) = 
    -(f(x)) \forall x \in \mathcal S$. \\
    
    Note that $\mathbb F^n$ and $\mathbb F^{\infty}$ are 
    special cases of $\mathbb F^{\mathcal S}$, 
    where $\mathcal S$ is a finite set of size $n$ or 
    an infinite set, respectively.
\end{example}

Note that the empty set $\phi$ is not a vector space, nor is it a subspace of any vector space.
\begin{example}
The set of differentiable real-valued functions is a subspace of $\mathbb R^{\mathbb R}$. Note that in calculus, the sum of two continuous functions 
is continuous, and the sum of two differentiable functions is differentiable. Also, scalar multiples of continuous and differentiable functions are continuous and differentiable, respectively.
\end{example}

\begin{definition}
    Let $V_1, \cdots V_n$ be subspaces of a vector space $\V$. Then, the sum of these subspaces is defined as 
    $$V_1 + V_2 + \cdots + V_n = \{ v_1 + v_2 + \cdots + v_n \mid v_i \in V_i \text{ for all } i \}$$
\end{definition}

\begin{example}
    Let $$V_1 = \left \{ (w, w, x, x) \in \mathbb F^4 | w, x \in \mathbb F \right \}$$
    $$V_2 = \left \{ (y, y, y, z) \in \mathbb F^4 | y, z \in \mathbb F \right \}$$
    Now, let $v_1 \in V_1$ and $v_2 \in V_2$. Then, we can write
    $$v_1 = (w_1, w_1, x_1, x_1)$$
    $$v_2 = (y_2, y_2, y_2, z_2)$$
    for some $w_1, x_1, y_2, z_2 \in \mathbb F$. Then, we have
    $$v_1 + v_2 = (w_1 + y_2, w_1 + y_2, x_1 + y_2, x_1 + z_2) \in V_1 + V_2$$
    Let $W$ be defined as 
    $$W = \left \{ (x, x, y, z) \in \mathbb F^4 | x, y, z, \in \mathbb F \right \}$$
    Then, $v_1 + v_2 \in W$ so $V_1 + V_2 \subseteq W$. \\
    Let $w \in W$. Then, we can write
    $$w = (x_w, x_w, y_w, z_w)$$
    for some $x_w, y_w, z_w \in \mathbb F$. Then, we have
    $$w = (x_w, x_w, y_w, z_w) = (x_w, x_w, y_w, y_w) + (0, 0, 0, z_w - y_w) \in V_1 + V_2$$
    $\therefore W = V_1 + V_2$
\end{example}

\begin{lemma}
    For any subspaces $V_1, \cdots V_n$ of a vector space $\V$, $V_1 + \cdots + V_n$ is a subspace of $\V$. It is also the smallest subspace of $V$
    that contains all elements of the form $v_1 + \cdots + v_n$ where $v_i \in V_i$ for all $i$.
\end{lemma}
\begin{proof}
    From the definition and that $V_1, \cdots V_n$ are subspaces,
    Since the subspaces themselves are closed under addition and scalar multiplication, $V_1 + \cdots + V_n$ is also closed under addition and scalar multiplication. Also, the zero vector $\0$ is in each of the subspaces, so $\0 \in V_1 + \cdots + V_n$. Thus, $V_1 + \cdots + V_n$ is a subspace of $\V$. 
\end{proof}

Note: Generally, the set theoretic union is rarely a subspace, except for trivial cases where one space is a subspace of the other. 
However, intersections of subspaces are generally subspaces.

\begin{definition}[Direct sum]
    Let $V_1, \cdots V_n$ be subspaces of a vector space $\V$. Then, the sum $V_1 + \cdots + V_n$ is called a direct sum if each element of $V_1 + \cdots + V_n$ can be written in one and only one way as $v_1 + \cdots + v_n$ where $v_i \in V_i$ for all $i$. 
    In this case, we say that the sum is a direct sum, denoted by 
    $$W = V_1 \oplus V_2 \oplus \cdots \oplus V_n$$
\end{definition}

\begin{example}
    Let $$U = \left \{ (x, x, y) \in \mathbb F^3 | x, y \in \mathbb F \right \}$$
    Let $$W = \left \{ (x, 0, 0) \in \mathbb F^3 | x \in \mathbb F \right \}$$
    Then, $U$ and $W$ are subspaces of $\mathbb F^3$.
    Any arbitrary vector in $\mathbb F^3$ can be written as
    $$\begin{pmatrix}
        a \\
        b \\
        c
    \end{pmatrix} = \begin{pmatrix}
        b \\
        b \\
        c
    \end{pmatrix} + \begin{pmatrix}
        a-b \\
        0 \\
        0
    \end{pmatrix}$$
    Since this is a unique representation, $U \oplus W = \mathbb F^3$.
\end{example}
\end{document}

